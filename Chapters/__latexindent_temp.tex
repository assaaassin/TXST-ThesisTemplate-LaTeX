In the past couple of years, IoT has made significant progress. From personal devices  such  as  smartwatches  and  environments  like  smart  homes,  to  industrially scaling  projects  like  smart  cities  and  smart,  autonomous  cars.There  has  been  an extensive  application  of  IoT  edge  devices  to  an  extent  that  today,  a  considerably significant amount of people possess at least one such device-be it your smart caror a smartwatch that you regularly use.Recent advances in the world ofIoT has increased the amount of use casesthat  these  personal  devices  such  as  smartwatches  have.But  having  additional functionality  also  poses  the  problem  of  increased  energy  consumption  on  devices already constrained by limited battery. For example, a device logging sensor data, say  temperature,  in  real-time  is  bound  to  use  more  energy  if  it  starts  logging  data from a motion sensor as well. This work studies the possibility of optimizing energy consumption for edge IoT devices by aiding the programmer in setting parameters that would ultimately lead   to   lesser   energy   consumption.   We   aim   to   focus   on   optimizing   the communication  cost  of  applications  by  determining  the  optimal  chunk  size,  with which the data should be transferred.It also involves a case study comparing energy consumption of native app and the same app running in a docker environment.There has been an extensive amount of work on energy optimization –both in  and  outside  of  IoT  environment.  Xiao  et  al.  [1]  compared  3G  and  Wi-Fi  while Balasubramanian  et  al.  [2]  compared  GSM,  3G  and  Wi-Fi  with  respect  to  energy consumption. However, neither of them compared energy consumption of respective data transfer technologies for throughput efficiency for the same task.Gupta et al. [3] made a measurement study of energy consumption when using VoIP  applications  with  Wi-Fi  connection  in  smartphones  and  showed  that  power save mode in Wi-Fi together with intelligent scanning techniques can reduce energy consumption. Xiao et al. [1] measured energy consumption for video streaming on mobile  apps  and  concluded  that  Wi-Fi  is  more  efficient  than  3G.  There  are  other measurement studies as well which compare different modes of communication but none of them discuss energy consumption differences for different packet sizes. The most related work is by Friedman et al. [4] where they measured power and throughput performance of Bluetooth and Wi-Fi usage in smartphones. They concluded that power consumption is generally linear with the obtained throughput, and Bluetooth uses less energy than Wi-Fi. However, this study also showed that different hardware (the device itself) and different software (the operating system for instance) have different results and there is no trend. It also concludes that an upper bound does exist bottlenecked bythe receiver after which the sender expends more power retransmitting the packets. This dependency of 
energy consumption on software is also discussed by Flinn and Satyanarayanan [5] who pointed out that “there is growing consensus that advances in battery technology and low-power circuit design cannot, by themselves, meet the energy needs of future mobile computers”[5]. This observation has been confirmed by recent advances in green software engineering, which demonstrated how the source of energy leaks can be software-related as well [6], [7], [8], [9].Our work on the other hand, considers multiple constraints in real-world IoT setting. Firstly, the transmission is not always continuous. Secondly, the amount of data to send depends on the amount of data collected by the sensors which may varybased on different applications. Thirdly, since the transmission is not continuous, if the app demands it, based off energy consumption data from different chunk sizes, an  optimal  communication  interval  canbe  set  to  reduce  the  amount  of  buffering required  (if  needed).  Lastly,  unique  apps  on  different  devices  with  different operating  systems  will  have  their  own  optimal  throughput  efficiency  for  energy consumption.  Given  the  hardware  configuration  of  such  device,  our  tool  can determine   the   optimal   chunk   size   for   data   transfer   with   respect   to   energy consumption.