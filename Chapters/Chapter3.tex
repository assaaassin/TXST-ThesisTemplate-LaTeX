\chapter{Related Work}
% \label{ch:relatedwork}
This section will discuss some of the related work to this thesis. There has been an extensive amount of work 
on energy optimization - both in and outside of IoT environment. Xiao et al. \cite{4756414} compared 3G and Wi-Fi 
technologies in terms of energy consumption. It answered some of the very first basic questions as to what are the 
factors that influence energy consumption? Therefore, it compared progressive download vs traditional TCP download-and-play, 
WCDMA vs WLAN 802.11g and phone memory vs external flash drive vs cache. It answered some important questions 
such as WLAN being more energy efficient than WCDMA, and the energy consumption of youtube video playback being 
dependent on the amount of video being loaded into cache. However, it failed to adapt to network conditions. Moreover, 
it was more about the comparison of resources used for video streaming as a whole rather than just the communication aspect of it. 
Our work on the other hand compares energy consumption under the same communication protocol i.e. TCP/IP for different 
chunk sizes. \\
Balasubramanian et al. \cite{confbal} compared GSM, 3G and Wi-Fi in terms of energy consumption and for each of them, 
developed a model for energy consumed by network activity. Using this model, they developed TailEnder, which was a 
protocol to reduce energy consumption for mobile phone apps. The main idea behind it was to schedule tasks that 
could tolerate a little bit of delay to reduce the overall cumulative energy expended. It also concluded that Wi-Fi 
was the most efficient of them all for different data transfer sizes as compared to 3G. However, it did not account for 
the differences between energy usage of different data transfer size within the same paradigm. Our work compares 
energy consumption of different data transfer sizes among other things. \\
Gupta et al. \cite{4292824} conducted a measurement study which provided a detailed anatomy of power consumption 
in WiFi phones that can be exploited in designing schemes to prolong the battery charge cycle. It also showed that 
power saving mode for Wifi together with intelligent scanning techniques can reduce overall energy consumption. 
The inference drawn from this work was that scanning algorithms have a significant impact on the battery life of 
WiFi enabled devices \cite{4292824}. However, this thesis is more geared towards optimizing the energy cost of 
communication after the connection has been established and data transfer needs to take place. \\