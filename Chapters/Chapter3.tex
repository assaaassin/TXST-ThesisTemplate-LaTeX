\chapter{Related Work}
% \label{ch:relatedwork}
This section will discuss some of the related work to this thesis. There has been an extensive amount of work 
on energy optimization - both in and outside of IoT environment. Xiao et al. \cite{4756414} compared 3G and Wi-Fi 
technologies in terms of energy consumption. It answered some of the very first basic questions as to what are the 
factors that influence energy consumption? Therefore, it compared progressive download vs traditional TCP download-and-play, 
WCDMA vs WLAN 802.11g and phone memory vs external flash drive vs cache. It answered some important questions 
such as WLAN being more energy efficient than WCDMA, and the energy consumption of youtube video playback being 
dependent on the amount of video being loaded into cache. However, it failed to adapt to network conditions. Moreover, 
it was more about the comparison of resources used for video streaming as a whole rather than just the communication aspect of it. 
Our work on the other hand compares energy consumption under the same communication protocol i.e. TCP/IP for different 
chunk sizes. \\
Balasubramanian et al. \cite{confbal} compared GSM, 3G and Wi-Fi in terms of energy consumption and for each of them, 
developed a model for energy consumed by network activity. Using this model, they developed TailEnder, which was a 
protocol to reduce energy consumption for mobile phone apps. The main idea behind it was to schedule tasks that 
could tolerate a little bit of delay to reduce the overall cumulative energy expended. It also concluded that Wi-Fi 
was the most efficient of them all for different data transfer sizes as compared to 3G. However, it did not account for 
the differences between energy usage of different data transfer size within the same paradigm. Our work compares 
energy consumption of different data transfer sizes among other things. \\
Gupta et al. \cite{4292824} conducted a measurement study which provided a detailed anatomy of power consumption 
in WiFi phones that can be exploited in designing schemes to prolong the battery charge cycle. It also showed that 
power saving mode for Wifi together with intelligent scanning techniques can reduce overall energy consumption. 
The inference drawn from this work was that scanning algorithms have a significant impact on the battery life of 
WiFi enabled devices \cite{4292824}. However, this thesis is more geared towards optimizing the energy cost of 
communication after the connection has been established and data transfer needs to take place. \\
Sahin et al. \cite{6224257} argued that even though historically, software developers leave concerns about power consumption 
to lower-level engineers, it is possible that involving software developers to participate in that process may result 
in more efficient applications. After detailed evaluation, they concluded that design pattern do impact energy consumption, and 
that impact within a category is inconsistent. It addressed the lack of tools and techniques to monitor power as an issue. Our 
work addresses the issue by presenting an energy estimation approach as well as builds upon their argument by helping 
software developers choose an optimal chunk size for a more energy efficient design process. \\
In another work, Sahin et al. \cite{10.1145/2652524.2652538} discuss the impact of code refactoring on energy usage. 
With the help of an empirical study, they found out that code refactoring can not only impact energy usage but also 
improve or deteriorate the energy efficiency of the whole program. The motivation of this thesis on the other hand is 
not really code refactoring but optimizing energy consumption of the WiFi adapter by figuring out a more optimal value for 
specific variables for certain API calls i.e. data transfer size for socket or HTTP APIs. \\
Samir et al. \cite{7886906} worked on energy consumption for different Java Collection Classes. They found 
out that different classes have different energy blueprint attached to them. In some cases, using an ArrayList is better than 
a LinkedList while in others, the former would incur a 300\% increase energy consumption due to unoptimized memory accesses 
and operations. Our work is more about what chunk size is better for which edge device so as to use the least amount of 
energy. Its more about studying what variable values in a program affect the hardware usage of a physical device in what manner. \\
The most related work is by Friedman et al. \cite{6200281} where they studied the effect of throughput on power in a 
performance study. They compared Bluetooth with WiFi and their usage in smartphones. The study was concluded with the result 
that power consumption is generally linear with the obtained throughput and overall, Bluetooth uses less energy than WiFi but 
ultimately its a trade-off between range and efficiency. It also showed that different hardware and different software produce 
different results for the same experiment, and there is no trend. One of the more interesting conclusions was that there 
exists an upper bound which is bottlenecked by the receiver (of throughput), when exceeded, would result in more power consumption than 
not. The additional consumption is explained by the fact that the sender has to retransmit the packets if the packet size is 
bigger than the receiving window. Our work on the other hand, is different in a sense that we build upon the conclusions of this work 
and show that even if there is no trend, there is a way to calculate the optimal chunk size i.e. throughput data size without 
knowing the detailed specs of the receiver. We also show that energy usage versus chunk size is also not linear. \\
Ngu et al. \cite{9446337} suggested that the use of an IoT middleware framework like Apache Cordova Host \cite{cordovahost} to overcome the drawbacks of cloud-based 
computing frameworks could help improve performance in terms of local processing. Along with the increased portability, the 
results showed that services deployed using the Cordova Accessor Host were 35\% more energy efficient than the native application. 
Even though our work focuses on reducing energy consumption as well, we focus more on individual variables that affect energy usage 
rather entire services. \\
In the next section, we will discuss the methodology employed to estimate energy consumption and determining optimal chunk size in 
our approach. 