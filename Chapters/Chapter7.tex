\chapter{Conclusion}

In this work, we have proposed a method for estimating energy consumption of edge IoT devices, as well as 
determining the optimal chunk size for the transmission of data over WiFi. Utilizing hardware specifications 
of the component of interest with the help of publicly available data sheets and combining the runtime information 
of applications, we estimate the total amount of energy consumed for communication. Energy-per-byte or EPB is then 
used to compute the energy consumption for a range of various chunk sizes. The chunk size corresponding to the 
lowest energy usage is the optimal chunk size for the transmission of data. \\

We evaluate our approach on two real-world applications of different natures. Both \texttt{TempSens} and the 
\texttt{Fall Detection} application have different use-cases, are deployed on different hardware devices and 
are developed in different languages. We show that our approach can improve energy utilization of edge IoT devices 
with regards to communication cost. We discuss the potential reasons why the \texttt{Fall Detection} application 
failed to work and we would like to fix the application issues to make it work in the near future. \\

We also present a case study where we discuss the possibility of using containers or Lingua Franca, to optimize 
energy consumption. Even though our evaluation of this was not extensive, we found out that there may be 
certain situations where the use of containers or Lingua Franca can improve efficiency in other aspects. For 
example, Docker may shine where maintainability and process isolation is important and the energy overhead it adds 
helps reduce other workloads such as costs for human-guided maintenance. Similarly, the use of Lingua Franca may 
help design better and efficient target programs in an easy to use and easy to understand language framework. It 
may not be the best option for energy optimization but it does help design distributed programs with fewer chances 
of concurrency bugs. \\

However, before we conclude, we would like to discuss some of the technical limitations that we encountered during 
our study. Firstly, for the Docker case study, we found out that the Docker container is not able to run on an 
Android device. This is due to the fact the Android kernel does not support Docker as it misses some of the key 
features, such as namespaces support, of the Linux kernel. Secondly, to the best of our knowledge, there is no 
container image of Android for Docker that operates on the ARM architecture. For these reasons alone, we were not 
able to conduct the Docker study for the Fall Detection application. We also wanted to compare communication cost via 
Bluetooth for \texttt{TempSens} but were unable to do due to \texttt{TempSensServer} being implemented on a 
MacOS. The bluetooth bindings for several bluetooth APIs don't properly work since MacOS drivers are proprietary 
and the best we could do was be able to scan for active devices. Moreover, Docker itself does not abstract and 
have a virtual bluetooth adapter. It directly uses the host's adapter and the host's bluetooth stack. The hardware 
device needs to be passed to the container much like the port. However, this means that bluetooth cost would 
theoretically be 
the same for the \texttt{TempSensClient} and \texttt{TempSensClientDocker} versions of the app.

For LF, it currently supports four target languages \textemdash Java, C, C++, and Python. Rust support is currently 
being added and Java support is currently being worked on. Since the current implementation of LF does not support 
Java, it is not possible to port the Fall Detection application to be implemented in LF. Whether we would 
get different results or not, the inability and limitation to test this in a much more conclusive manner is a 
hinderance that we were not able to overcome simply because its not supported yet. \\

To summarize, this thesis presents an energy estimation and optimal chunk size calculation method for 
edge IoT devices. We show that this take on energy optimization has not been proposed before and evaluate it in 
a real-world setting. Our results show that this approach can improve energy utilization of edge IoT devices in 
terms of communication cost. In the future, we would like to adapt the optimal chunk size calculation script 
to automatically compute the estimated power consumption of the wifi adapter given a manufacturer and 
respective data sheet. For example, for Qualcomm WCN3620 we could parse tables corresponding to \textquotedblleft Power Consumption\textquotedblright 
 and for Broadcom BCM43455, we could parse tables corresponding to \textquotedblleft WLAN Current Consumption\textquotedblright. 
 This is due to the fact that manufacturers follow their own standards for providing data sheets, and 
 consequently, all Broadcom data sheets will provide the current information under \textquotedblleft WLAN Current Consumption\textquotedblright, 
 and same goes for other manufacturers. This way we 
have an automated way of estimating the power consumption of an edge device's hardware as well as calculating 
the optimal chunk size in a single script. Other than that, we would also like 
to calculate the optimal chunk size for Bluetooth communication and analyze the results. 