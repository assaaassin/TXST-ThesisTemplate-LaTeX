\addtocontents{toc}{\cftpagenumbersoff{part}}
\addcontentsline{toc}{part}{CHAPTER}

\chapter{Introduction}\label{ch:chapter1}

In the past couple of years, IoT has made significant progress. From personal devices 
such as smartwatches and environment like smart homes, to industrially scaling projects 
like smart cities and smart, autonomous cars. There has been an extensive application of 
IoT edge devices to an extent that today, a considerably significant amount of people possess 
at least one such device \textemdash be it your smart car or a smartwatch that you regularly 
or routinely use. \\
Recent advances in the world of IoT have increased the amount of use cases that these 
personal devices such as smartwatches have. But having additional functionality also poses 
the problem of increased energy consumption on devices already constrained by limited battery. 
For example, a device logging sensor data, say temperature, in real-time is bound to use 
more energy if it starts logging data from a motion sensor as well. \\
This work studies the possibility of optimizing energy consumption for edge IoT devices 
by aiding the programmer in setting parameters that would ultimately lead to lesser energy 
consumption. We aim to focus on optimizing the communication cost of applications by determining 
the optimal chunk size, with which the data should be transferred. It also involves a case study 
comparing energy consumption of native app and the same app running in a container environment 
such as Docker (cite here). \\
There has been an extensive amount of work on energy optimization \textemdash both without 
and in an IoT setting. Xiao et al. \cite{4756414} compared 3G and Wi-fi while Balasubramanian 
et al. (cite here) compared GSM, 3G and Wi-Fi with respect to energy consumption. However, 
neither of them compared energy consumption of respective data transfer technologies 
for throughput efficiency for the same task. \\