\addtocontents{toc}{\cftpagenumbersoff{part}}
\addcontentsline{toc}{part}{CHAPTER}

\chapter{Introduction}\label{ch:chapter1}

In the past couple of years, IoT has made significant progress. From personal devices 
such as smartwatches and environment like smart homes, to industrially scaling projects 
like smart cities and smart, autonomous cars. There has been an extensive application of 
IoT edge devices to an extent that today, a considerably significant amount of people possess 
at least one such device \textemdash be it your smart car or a smartwatch that you regularly 
or routinely use. \\
Recent advances in the world of IoT have increased the amount of use cases that these 
personal devices such as smartwatches have. But having additional functionality also poses 
the problem of increased energy consumption on devices already constrained by limited battery. 
For example, a device logging sensor data, say temperature, in real-time is bound to use 
more energy if it starts logging data from a motion sensor as well. \\
This work studies the possibility of optimizing energy consumption for edge IoT devices 
by aiding the programmer in setting parameters that would ultimately lead to lesser energy 
consumption. We aim to focus on optimizing the communication cost of applications by determining 
the optimal chunk size, with which the data should be transferred. It also involves a case study 
comparing energy consumption of native app and the same app running in a container environment 
such as Docker (cite here). \\
There has been an extensive amount of work on energy optimization \textemdash both without 
and in an IoT setting. Xiao et al. \cite{4756414} compared 3G and Wi-fi while Balasubramanian 
et al. (cite here) compared GSM, 3G and Wi-Fi with respect to energy consumption. However, 
neither of them compared energy consumption of respective data transfer technologies 
for throughput efficiency for the same task. \\
Gupta et al. (cite here) made a measurement study of energy consumption when using VoIP applications 
with Wi-Fi connection in smartphones and showed that power saving mode in Wi-Fi together with 
intelligent scanning techniques can reduce energy consumption. Xiao et al. (cite here) measured 
energy consumption for video streaming on mobile apps and concluded that Wi-Fi is more 
efficient than 3G. There are other measurement studies as well which compare different modes 
of communication but none of them discuss energy consumption differences for different packet 
sizes. \\
The work that seems the most related to ours is by Friedman et al. (cite here) where they measured 
power and throughput performance of Bluetooth and Wi-Fi usage in smartphones. They concluded that 
power consumption is generally linear with the obtained throughput, and Bluetooth uses less 
energy than Wi-Fi. However, this study also showed that different hardware and different software 
have different results and there is no generic trend. They also concluded that an upper bound 
does exist, bottlenecked by the receiver, after which the sender expends more power 
retransmitting packets. This dependency of energy consumption on software is also discussed by 
Flinn and Satyanarayanan (cite here) who point out that \textquotedblleft \textit{There is growing 
consensus that advances in battery technology and low-power circuit design cannot, by themselves, 
meet the energy needs of future movile computers} \textquotedblright (cite here). This observation 
has been confirmed by recent advances in green software engineering, which demonstrated how the 
source of energy leaks can be software-related as well (cite here). \\
Our work on the other hand, considers multiple constraints in real-world IoT setting. Firstly, 
the transmission is not always continuous. The data may be communicated in regular or irregular 
intervals. Secondly, the amount of data to send depends on the amount of data collected by 
the sensors which may vary based on different applications. Thirdly, since the transmission may 
not be continuous, if the app demands it, based off energy consumption data from different chunk 
sizes, an optimal communication interval can be set to reduce the amount of buffering required 
(if needed). Lastly, unique apps on different devices with different operating systems will have 
their own optimal throughput efficiency for energy consumption. Given the hardware configuration 
of such a device, our tool can determine the optimal chunk size for data transfer with respect 
to energy consumption. \\
\subsection{Contributions}
This thesis has the following contributions. It: 
\begin{itemize}
    \item proposes a method to estimate energy consumption of edge IoT devices 
    \item proposes a new approach to reduce energy consumption with the help of optimal chunk size
    \item presents a new metric called Energy-Per-Byte or EPB
    \item discusses a case study to compare energy consumption between native, Lingua Franca 
    and container version of the same app.
    \item evaluates the effectiveness of this approach on two real-world applications of different domains.
\end{itemize}

The organization of the rest of the thesis as as follows. In chapter II, we will provide 
background information about edge IoT devices, energy usage and estimation, containers and middlware 
accessor frameworks. In chapter III, we briefly sumarize the related work and chapter IV will discuss 
our methodology in detail. In chapter V, we will discuss our evaluation and results, and provide 
our reasoning. Chapter VI, will discuss the case study of energy cost comparison between native, 
Lingua Franca and Docker version of the same application and finally in chapter VII, we will conclude 
our thesis and discuss possible future work.
