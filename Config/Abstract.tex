%*******************************************************
% Abstract
%*******************************************************
\phantomsection
\mychapter{0}{ABSTRACT}
\begin{center}
    \textbf{ABSTRACT} 
\end{center}

Recent advances in the world of IoT have significantly improved what one 
can do with the help of personal gadgets such as smartwatches and other 
edge devices. This improvement has lead to the inclusion of additional 
functionality like fall detection, temperature and motion sensing etcetra. 
Although this in itself is not a bad thing, however, this does pose the 
problem of increased energy consumption on devices which are already 
constrained by limited battery. We present our study which discusses the 
possibility of optimizing energy consumption for edge IoT devices by reducing 
communication cost of applications by determining an optimal chunk size. We 
present a metric \textbf{energy-per-byte} or \textbf{EPB} which helps determine 
the optimal chunk size for respective edge devices. We evaluate our approach 
with the help of two case studies using different edge devices and applications, 
and we achieve considerable energy optimization depending on the app functionality 
and the hardware involved.