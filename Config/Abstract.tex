%*******************************************************
% Abstract
%*******************************************************
\phantomsection
\mychapter{0}{ABSTRACT}
\begin{center}
    \textbf{ABSTRACT} 
\end{center}

Recent advances in the world of IoT have significantly improved what one 
can do with the help of personal gadgets such as smartwatches and other 
edge devices. This improvement has lead to the inclusion of additional 
functionality like fall detection, temperature and motion sensing etc. using an edge device. 
Although this in itself is not a bad thing, however, this does pose the 
problem of increased energy consumption on devices which are already 
constrained by limited battery. A portion of this energy is consumed during client-server communication 
which is an essential part of applications that either perform major computations from sensor data on the cloud 
or use the server to backup user data. 
We present our study which discusses the 
possibility of optimizing energy consumption for edge IoT devices by reducing 
communication cost of applications by determining an optimal chunk size. We 
present a metric \textbf{energy-per-byte} or \textbf{EPB} which helps determine 
the optimal chunk size for respective edge devices. We also discuss a case study to (1) determine 
whether containerized applications are energy efficient and (2) if a coordination language 
framework like Lingua Franca can help design energy efficient applications.
We evaluate our approach 
with the help of two applications running on different edge devices and softwares, 
and we show that we achieve considerable energy optimization depending on the application functionality 
and the hardware involved.